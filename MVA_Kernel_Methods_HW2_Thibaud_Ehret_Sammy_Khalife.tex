%%%%%%%%%%%%%%%%%%%%%%%%%%%%%%%%%%%%%%%%%
% University Assignment Title Page 
% LaTeX Template
% Version 1.0 (27/12/12)
%
% This template has been downloaded from:
% http://www.LaTeXTemplates.com
%
% Original author:
% WikiBooks (http://en.wikibooks.org/wiki/LaTeX/Title Creation)
%
% License:
% CC BY-NC-SA 3.0 (http://creativecommons.org/licenses/by-nc-sa/3.0/)
% 
% Instructions for using this template:
% This title page is capable of being compiled as is. This is not useful for 
% including it in another document. To do this, you have two options: 
%
% 1) Copy/paste everything between \begin{document} and \end{document} 
% starting at \begin{titlepage} and paste this into another LaTeX file where you 
% want your title page.
% OR
% 2) Remove everything outside the \begin{titlepage} and \end{titlepage} and 
% move this file to the same directory as the LaTeX file you wish to add it to. 
% Then add \input{./title page_1.tex} to your LaTeX file where you want your
% title page.
%
%%%%%%%%%%%%%%%%%%%%%%%%%%%%%%%%%%%%%%%%%

%----------------------------------------------------------------------------------------
%	PACKAGES AND OTHER DOCUMENT CONFIGURATIONS
%----------------------------------------------------------------------------------------

\documentclass[12pt]{article}
\usepackage{graphicx}
\usepackage[utf8]{inputenc}  
\usepackage[T1]{fontenc} 
\usepackage[top=1cm,bottom=1cm,left=0.5cm,right=1.5cm,asymmetric]{geometry}
\usepackage{amsfonts}
\usepackage{graphicx}
\usepackage{amsmath}
\usepackage{caption}
\usepackage{subcaption}
\usepackage{float}
\usepackage{subfig}
\usepackage{fancyhdr}
\pagestyle{fancy}
\renewcommand{\footrulewidth}{1pt}
\fancyhead[R]{\textit{Master MVA : Kernel Methods}}
\fancyfoot[L]{\textit{}}
%\usepackage{unicode-math}
%\setmathfont{XITS Math}
%\setmathfont[version=setB,StylisticSet=1]{XITS Math}
\usepackage{array,multirow,makecell}
\setcellgapes{1pt}
\makegapedcells
\newcolumntype{R}[1]{>{\raggedleft\arraybackslash }b{#1}}
\newcolumntype{L}[1]{>{\raggedright\arraybackslash }b{#1}}
\newcolumntype{C}[1]{>{\centering\arraybackslash }b{#1}}

\pagestyle{fancy}
\renewcommand{\footrulewidth}{1pt}
\fancyfoot[L]{\textit{}}
\newcommand{\cond}{(x_i|x_{\pi_i})}

%\usepackage{caption}
%\usepackage{subcaption}


%\usepackage{unicode-math}
%\setmathfont{XITS Math}
%\setmathfont[version=setB,StylisticSet=1]{XITS Math}


%\geometry{hmargin=1.5cm,vmargin=2cm}   

\geometry{hmargin=2.5cm,vmargin=2cm}   
\begin{document}

\section*{Kernel Methods : Homework 2}
\section*{Thibaud Ehret \& Sammy Khalife}
\subsubsection*{01/02/2014}

\section*{1)}
The formula for the projection on the $i^\text{th}$ eigenvector is $ \sum_{j=1}^{n} \alpha^{(i)}_j \Phi(x_j) $.
Therefore after injecting into the expression of $\Psi$,
\begin{eqnarray*}
	\Psi(x) &=& \sum_{i=1}^d \langle \sum_{j=1}^{n} \alpha^{(i)}_j \Phi(x_j), \Phi(x) - m\rangle \left(\sum_{j=1}^{n} \alpha^{(i)}_j \Phi(x_j)\right)+ m\\
	&=& \sum_{i=1}^d \left(\sum_{j=1}^{n} \alpha^{(i)}_j \langle \Phi(x_j), \Phi(x) \rangle\right) \left(\sum_{j=1}^{n}\alpha^{(i)}_j \Phi(x_j)\right) - \sum_{i=1}^d \left(\sum_{j=1}^{n} \alpha^{(i)}_j \langle \Phi(x_j), m \rangle\right)\left(\sum_{j=1}^{n}\alpha^{(i)}_j \Phi(x_j)\right)+ m\\
	&=& \sum_{i=1}^d \left(\sum_{j=1}^{n} \alpha^{(i)}_j K(x_j,x)\right) \left(\sum_{j=1}^{n}\alpha^{(i)}_j \Phi(x_j)\right) - \sum_{i=1}^d \left(\sum_{j=1}^{n} \alpha^{(i)}_j \langle \Phi(x_j), \frac{1}{n}\sum_{u=1}^{n} \Phi(x_u) \rangle\right) \left(\sum_{j=1}^{n}\alpha^{(i)}_j \Phi(x_j)\right)\\
	&&+ \frac{1}{n}\sum_{u=1}^{n} \Phi(x_u)\\
	&=& \sum_{i=1}^d \left(\sum_{j=1}^{n} \alpha^{(i)}_j K(x_j,x)\right) \left(\sum_{j=1}^{n}\alpha^{(i)}_j \Phi(x_j)\right) - \sum_{i=1}^d \left(\sum_{j=1}^{n} \alpha^{(i)}_j \frac{1}{n}\sum_{u=1}^{n} K(x_j,x_u)\right) \left(\sum_{j=1}^{n}\alpha^{(i)}_j \Phi(x_j)\right)\\
	&&+ \frac{1}{n}\sum_{u=1}^{n} \Phi(x_u)\\
	&=& \sum_{i=1}^d \left(\sum_{j=1}^{n} \alpha^{(i)}_j K(x_j,x)\right) \left(\sum_{j=1}^{n}\alpha^{(i)}_j \Phi(x_j)\right) - \sum_{i=1}^d \left(\sum_{j=1}^{n} \alpha^{(i)}_j \frac{1}{n}\sum_{u=1}^{n} K(x_j,x_u)\right) \left(\sum_{j=1}^{n}\alpha^{(i)}_j \Phi(x_j)\right)\\
	&&+ \frac{1}{n}\sum_{u=1}^{n} \Phi(x_u)\\
	&=& \sum_{j=1}^n \left( \sum_{u=1}^{n} \left(\sum_{i=1}^{d}\alpha^{(i)}_u \alpha^{(i)}_j\right) K(x_u,x) - \sum_{u=1}^{n} \left(\sum_{i=1}^{d}\alpha^{(i)}_u \alpha^{(i)}_j\right) \left(\frac{1}{n} \sum_{v=1}^n K(x_u,x_v)\right) + \frac{1}{n} \right) \Phi(x_j)
\end{eqnarray*}

Therefore 
\[\gamma_j = \sum_{u=1}^{n} \left(\sum_{i=1}^{d}\alpha^{(i)}_u \alpha^{(i)}_j\right) K(x_u,x) - \sum_{u=1}^{n} \left(\sum_{i=1}^{d}\alpha^{(i)}_u \alpha^{(i)}_j\right) \left(\frac{1}{n} \sum_{v=1}^n K(x_u,x_v)\right) + \frac{1}{n} \]
\section*{2)}

\begin{eqnarray*}
	f(y) &=& \|\Phi(y) - \Psi(x) \|^2\\
	&=& \langle \Phi(y) - \Psi(x), \Phi(y) - \Psi(x) \rangle \\
	&=& \langle \Phi(y), \Phi(y) \rangle -  2 \langle \Phi(y), \Psi(x) \rangle + \langle \Psi(x), \Psi(x) \rangle\\
	&=& K(y,y) -  2 \langle \Phi(y), \Psi(x) \rangle + \langle \Psi(x), \Psi(x) \rangle
\end{eqnarray*}

Using the the fact that $\Psi(x) = \sum_{i=1}^n \gamma_i \Phi(x_i)$,
\begin{eqnarray*}
	f(y) &=& K(y,y) - 2\langle \Phi(y), \sum_{i=1}^n \gamma_i \Phi(x_i \rangle + \langle \sum_{i=1}^n \gamma_i \Phi(x_i), \sum_{i=1}^n \gamma_i \Phi(x_i) \rangle \\
	&=& K(y,y) - 2\sum_{i=1}^n \gamma_i \langle \Phi(y), \Phi(x_i) \rangle + \sum_{i=1}^n \sum_{j=1}^n \gamma_i\gamma_j\langle  \Phi(x_i),  \Phi(x_j) \rangle \\
	&=& K(y,y) - 2\sum_{i=1}^n \gamma_i K(y,x_i) + \sum_{i=1}^n \sum_{j=1}^n \gamma_i\gamma_j K(x_i,x_j)
\end{eqnarray*}

\end{document}
